%!TEX root = main.tex
\section{Case beskrivelse}
Hver måned får kunderne udsendt bestillingssedler på papirform, hvor de ved afkrydsning kan bestemme en måned frem, hvad de kunne tænke sig at spise.
Disse bestillingssedler bliver så sendt tilbage til køkkenet, hvor en medarbejder går igennem ca. 2000 \todo{indsæt kilde} kunders madønsker, tæller op hvor mange af hver menu, der skal produceres og taster det derefter manuelt ind i deres nuværende system.
Dette er der sat 14 dage af til, hvilket til tider ikke er nok, hvilket resulterer i, at personen, der køber ingredienser, er nødsaget til at estimere antal bestillinger af hver ret og at køkkenpersonalet dagligt må checke de præcise tal for dagens madlavning.
Der kan også opstå det problem, at medarbejderen overser en eller flere bestillinger, hvilket leder til manglende leveringer.
Kunden er så nødsaget til at tage kontakt til køkkenet, så køkkenarbejderne manuelt skal indskrive det i deres system.
Derfor ønsker Madservice Aalborgs køkkener en bedre måde at indsamle kundernes madønsker til deres nuværende IT system.